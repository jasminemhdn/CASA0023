% Options for packages loaded elsewhere
\PassOptionsToPackage{unicode}{hyperref}
\PassOptionsToPackage{hyphens}{url}
\PassOptionsToPackage{dvipsnames,svgnames,x11names}{xcolor}
%
\documentclass[
  letterpaper,
  DIV=11,
  numbers=noendperiod]{scrreprt}

\usepackage{amsmath,amssymb}
\usepackage{iftex}
\ifPDFTeX
  \usepackage[T1]{fontenc}
  \usepackage[utf8]{inputenc}
  \usepackage{textcomp} % provide euro and other symbols
\else % if luatex or xetex
  \usepackage{unicode-math}
  \defaultfontfeatures{Scale=MatchLowercase}
  \defaultfontfeatures[\rmfamily]{Ligatures=TeX,Scale=1}
\fi
\usepackage{lmodern}
\ifPDFTeX\else  
    % xetex/luatex font selection
\fi
% Use upquote if available, for straight quotes in verbatim environments
\IfFileExists{upquote.sty}{\usepackage{upquote}}{}
\IfFileExists{microtype.sty}{% use microtype if available
  \usepackage[]{microtype}
  \UseMicrotypeSet[protrusion]{basicmath} % disable protrusion for tt fonts
}{}
\makeatletter
\@ifundefined{KOMAClassName}{% if non-KOMA class
  \IfFileExists{parskip.sty}{%
    \usepackage{parskip}
  }{% else
    \setlength{\parindent}{0pt}
    \setlength{\parskip}{6pt plus 2pt minus 1pt}}
}{% if KOMA class
  \KOMAoptions{parskip=half}}
\makeatother
\usepackage{xcolor}
\setlength{\emergencystretch}{3em} % prevent overfull lines
\setcounter{secnumdepth}{5}
% Make \paragraph and \subparagraph free-standing
\makeatletter
\ifx\paragraph\undefined\else
  \let\oldparagraph\paragraph
  \renewcommand{\paragraph}{
    \@ifstar
      \xxxParagraphStar
      \xxxParagraphNoStar
  }
  \newcommand{\xxxParagraphStar}[1]{\oldparagraph*{#1}\mbox{}}
  \newcommand{\xxxParagraphNoStar}[1]{\oldparagraph{#1}\mbox{}}
\fi
\ifx\subparagraph\undefined\else
  \let\oldsubparagraph\subparagraph
  \renewcommand{\subparagraph}{
    \@ifstar
      \xxxSubParagraphStar
      \xxxSubParagraphNoStar
  }
  \newcommand{\xxxSubParagraphStar}[1]{\oldsubparagraph*{#1}\mbox{}}
  \newcommand{\xxxSubParagraphNoStar}[1]{\oldsubparagraph{#1}\mbox{}}
\fi
\makeatother


\providecommand{\tightlist}{%
  \setlength{\itemsep}{0pt}\setlength{\parskip}{0pt}}\usepackage{longtable,booktabs,array}
\usepackage{calc} % for calculating minipage widths
% Correct order of tables after \paragraph or \subparagraph
\usepackage{etoolbox}
\makeatletter
\patchcmd\longtable{\par}{\if@noskipsec\mbox{}\fi\par}{}{}
\makeatother
% Allow footnotes in longtable head/foot
\IfFileExists{footnotehyper.sty}{\usepackage{footnotehyper}}{\usepackage{footnote}}
\makesavenoteenv{longtable}
\usepackage{graphicx}
\makeatletter
\newsavebox\pandoc@box
\newcommand*\pandocbounded[1]{% scales image to fit in text height/width
  \sbox\pandoc@box{#1}%
  \Gscale@div\@tempa{\textheight}{\dimexpr\ht\pandoc@box+\dp\pandoc@box\relax}%
  \Gscale@div\@tempb{\linewidth}{\wd\pandoc@box}%
  \ifdim\@tempb\p@<\@tempa\p@\let\@tempa\@tempb\fi% select the smaller of both
  \ifdim\@tempa\p@<\p@\scalebox{\@tempa}{\usebox\pandoc@box}%
  \else\usebox{\pandoc@box}%
  \fi%
}
% Set default figure placement to htbp
\def\fps@figure{htbp}
\makeatother
% definitions for citeproc citations
\NewDocumentCommand\citeproctext{}{}
\NewDocumentCommand\citeproc{mm}{%
  \begingroup\def\citeproctext{#2}\cite{#1}\endgroup}
\makeatletter
 % allow citations to break across lines
 \let\@cite@ofmt\@firstofone
 % avoid brackets around text for \cite:
 \def\@biblabel#1{}
 \def\@cite#1#2{{#1\if@tempswa , #2\fi}}
\makeatother
\newlength{\cslhangindent}
\setlength{\cslhangindent}{1.5em}
\newlength{\csllabelwidth}
\setlength{\csllabelwidth}{3em}
\newenvironment{CSLReferences}[2] % #1 hanging-indent, #2 entry-spacing
 {\begin{list}{}{%
  \setlength{\itemindent}{0pt}
  \setlength{\leftmargin}{0pt}
  \setlength{\parsep}{0pt}
  % turn on hanging indent if param 1 is 1
  \ifodd #1
   \setlength{\leftmargin}{\cslhangindent}
   \setlength{\itemindent}{-1\cslhangindent}
  \fi
  % set entry spacing
  \setlength{\itemsep}{#2\baselineskip}}}
 {\end{list}}
\usepackage{calc}
\newcommand{\CSLBlock}[1]{\hfill\break\parbox[t]{\linewidth}{\strut\ignorespaces#1\strut}}
\newcommand{\CSLLeftMargin}[1]{\parbox[t]{\csllabelwidth}{\strut#1\strut}}
\newcommand{\CSLRightInline}[1]{\parbox[t]{\linewidth - \csllabelwidth}{\strut#1\strut}}
\newcommand{\CSLIndent}[1]{\hspace{\cslhangindent}#1}

\KOMAoption{captions}{tableheading}
\makeatletter
\@ifpackageloaded{bookmark}{}{\usepackage{bookmark}}
\makeatother
\makeatletter
\@ifpackageloaded{caption}{}{\usepackage{caption}}
\AtBeginDocument{%
\ifdefined\contentsname
  \renewcommand*\contentsname{Table of contents}
\else
  \newcommand\contentsname{Table of contents}
\fi
\ifdefined\listfigurename
  \renewcommand*\listfigurename{List of Figures}
\else
  \newcommand\listfigurename{List of Figures}
\fi
\ifdefined\listtablename
  \renewcommand*\listtablename{List of Tables}
\else
  \newcommand\listtablename{List of Tables}
\fi
\ifdefined\figurename
  \renewcommand*\figurename{Figure}
\else
  \newcommand\figurename{Figure}
\fi
\ifdefined\tablename
  \renewcommand*\tablename{Table}
\else
  \newcommand\tablename{Table}
\fi
}
\@ifpackageloaded{float}{}{\usepackage{float}}
\floatstyle{ruled}
\@ifundefined{c@chapter}{\newfloat{codelisting}{h}{lop}}{\newfloat{codelisting}{h}{lop}[chapter]}
\floatname{codelisting}{Listing}
\newcommand*\listoflistings{\listof{codelisting}{List of Listings}}
\makeatother
\makeatletter
\makeatother
\makeatletter
\@ifpackageloaded{caption}{}{\usepackage{caption}}
\@ifpackageloaded{subcaption}{}{\usepackage{subcaption}}
\makeatother

\usepackage{bookmark}

\IfFileExists{xurl.sty}{\usepackage{xurl}}{} % add URL line breaks if available
\urlstyle{same} % disable monospaced font for URLs
\hypersetup{
  pdftitle={Learning Diary - CASA0023: Remotely Sensing Cities and Environment},
  pdfauthor={Jasmine Mahdani},
  colorlinks=true,
  linkcolor={blue},
  filecolor={Maroon},
  citecolor={Blue},
  urlcolor={Blue},
  pdfcreator={LaTeX via pandoc}}


\title{Learning Diary - CASA0023: Remotely Sensing Cities and
Environment}
\author{Jasmine Mahdani}
\date{}

\begin{document}
\maketitle

\renewcommand*\contentsname{Table of contents}
{
\hypersetup{linkcolor=}
\setcounter{tocdepth}{2}
\tableofcontents
}

\bookmarksetup{startatroot}

\chapter*{Personal Introduction}\label{personal-introduction}
\addcontentsline{toc}{chapter}{Personal Introduction}

\markboth{Personal Introduction}{Personal Introduction}

My name is Jasmine Mahdani and I am from Indonesia. I graduated from ITB
(Bandung Institute of Technology) in 2022, majoring in Geodesy and
Geomatic Engineering. During my undergraduate years, I had been actively
engaged in research particularly for urban related issues using GIS and
remote sensing. My research topic focus on water and sanitation access,
slum identification, and disaster risk assessment. My enthusiasm to
urban issues and GIS continues after I graduate that leads me to choose
urban spatial researcher as my career path. However, doing spatial
analysis for urban issues is quite tricky because the coverage area is
relatively small and finding open source detailed data set in Indonesia
is extremely hard. Moreover, I also have limited knowledge on how urban
system works. Therefore, I decided to pursue master degree in Urban
Spatial Science at UCL (University College London) to deepen my
knowledge in urban analytics and to support my career path as a
researcher. In Urban Spatial Science, I take the Urban Modelling and
Simulation pathway and Remotely Sensing Cities and Environment modules
in hope I can answer the challenge of using open sourced medium spatial
resolution remote sensing data for urban research that mostly have small
coverage area.

\bookmarksetup{startatroot}

\chapter{Introduction to Remote
Sensing}\label{introduction-to-remote-sensing}

\section{Summary: Remote Sensing Definition and How it
Works}\label{summary-remote-sensing-definition-and-how-it-works}

Remote sensing is the practice of obtaining information about the
Earth's surface, using images acquired from airborne or spaceborne
vehicles by measuring reflected, emitted, or returned electromagnetic
radiation (Ruth DeFries, 2013). The objective of this technology is to
provide observation of physical parameter in a mapping frame at a given
time period (Toth \& Jozkow, 2016).

Remote sensing captures the electromagnetic (EM) radiation that is
reflected from earth's surface. EM wave is an energy that travels
through the light. All matters with absolute temperature above zero
reflect and emit EM waves of various length (J. M. Read and M. Torrado,
2009). A material that fully capable of absorbing and re-emitting all EM
energy that it receives is called a blackbody, but it is rare and most
natural objects only absorb some of the energy (Tempfli K. et al.,
2009). The difference in how objects absorb and reflect energy is what's
make every earth's feature has its own signature characteristic which is
recorded as digital number (DN) in each pixel of images using the
sensors.

In remote sensing, there are two type of sensors, passive and active
that are used for different applications. Most sensors use passive
systems which can only capture the reflection of sun's energy and work
during daylight. Therefore, this sensors' orbit sometimes are set to
follow's the sun (sun-synchronous). The electromagnetic spectrum
recorded by the sensors range from ultraviolet (UV), visible (red,
green, blue), infrared (IR), and microwave. Meanwhile, active remote
sensing can radiate its own energy and measure the amount of radiation
returned to the sensor. Active sensors can penetrate through clouds and
won't be affected by daylight or weather conditions. Example of active
sensors included SAR and LIDAR.

Although remote sensing is great for collecting data in inaccessible
area and cheaper for mapping large areas, there are 4 type of resolution
that needs to be considered when using it: spatial, spectral, temporal,
and radiometric. Spatial resolution refers to the size of area measured
which is represented by each pixel. Spectral refer to the range of
wavelength the sensor is sensitive to. Radiometric refer to the
difference in radiation intensity. Temporal characteristics refer to the
time of image acquisition. These resolutions need to be considered based
on the purpose of the mapping.

\section{Application: Example of Landsat and Sentinel for Urban Heat
Island
(UHI)}\label{application-example-of-landsat-and-sentinel-for-urban-heat-island-uhi}

With the growing technology, remote sensing becomes more advanced and
freely available. The development of single sensor to multiple sensors
allows spatial researcher to do environmental and social research using
it. The example of application includes biodiversity monitoring, crop
classification, hazards modelling, Urban Heat Island monitoring, LULC
classification, and carbon sequestration modelling (Roy et al., 2017).
Sometimes people compare and combine multiple sensors from different
platform to get accurate and comprehensive results because different
sensors capturing the same area might get different reflectance values.
Examples of known satellites for research are Terra, MODIS, SPOT,
Landsat, and Sentinel (Toth \& Jozkow, 2016).

The paper by Rech et al., (2024) gives example of using Landsat 8 TIRS
band for mapping Surface Urban Heat Islands (SUHI) and assess the SUHI
relation to albedo, elevation, land surface emissivity, and vegetation
cover across Local Climate Zones (LCZ) in Florianopolis, Brazil. The
paper successfully assessing the variability of SUHI between day and
night, different LCZ and different surface using multiple statistical
approaches. Although the paper mentioned using Practical Single-Channel
(PSC) algorithm to derive LST, the result does not explain the accuracy
of the methods or the LST before used for analysis and instead only
focuses on the variability with the parameters. Since the study area is
a humid region, the use of PSC method give concern on its reliability.
However, I think the author has tried it best to find images with low
cloud cover and no rain. Unfortunately, there's only pair that meets
these criteria, so it's not feasible to conduct the study for different
seasons. Moreover, since the TIRS band already resampled to 30 meters, I
think it would be good if they zoom in to some specific areas to show
variability.

Meanwhile, the paper by García and Díaz (2021) used Sentinel-3 to
Meanwhile, the paper by García and Díaz (2021) used Sentinel-3 to derive
LST and applied Split-Window Correction for assessing Urban Heat Island
and factors contributing to it in Granada, Spain. Different from
previous one, this study compares the value derived from Sentinel-3 with
in-situ analysis. The result shows Sentinel-3 temperature is a little
bit higher. The study also conduct analysis for different seasons
because Sentinel-3 temporal resolution is daily giving more available
data to used. However, the Sentinel-3 1 km resolution might not be as
detailed as Landsat 8, even though the author had tried to resample it
to 100 meters.

\section{Reflection: Challenges in Remote Sensing and Potential Future
Study}\label{reflection-challenges-in-remote-sensing-and-potential-future-study}

Since I have been learning about remote sensing on my undergraduate
degree before, the material about sensors and electromagnetic
reflectance are already familiar for me. I also have used some remote
sensing data for my research and since the first time I know about it, I
always think remote sensing mapping is a cheaper method for large area
compared to other mapping techniques. However, when I did the practical
using SNAP for Sentinel-2 and Landsat-8 data, I started to think even
though the medium spatial resolution remote sensing images are freely
available, we still need high specification computer to process it
because one tiles comprised of several bands are huge and my laptop
somewhat need to work hard to do the resampling and other practical. It
also made me realise that it's not quite efficient to do preprocessing
of these images from scratch when you only want to do research on very
small area especially when the boundary does not overlap perfectly with
the image. I think using Google Earth Engine (GEE) which would be
explained in the later weeks is better because they provide some already
pre-processed images and its based-on cloud, so you don't need high
specification computer. Beside this issue, I also think that this
`cheaper method' only applies to regional-based analysis that can use
medium spatial resolution like Landsat and Sentinel-2. Meanwhile, if you
want to do a detailed analysis on city like how the UHI effect on
different surface and LCZ, especially for building level, it will need
high spatial resolution which are mostly still commercial. The
resampling methods may give us immediate solution for this although we
still need to consider the impact of each method that may affect the
analysis. However, despite all the challenges in remote sensing
approaches, I still think it's the most robust method in the present
that can give comprehensive result when its used and combined with other
dataset appropriately. To this day, I still have a goal to do UHI
analysis on building level using open-source remote sensing data. I
believe remote sensing technology would develop even further that gives
more better resolution from the existing ones.

\bookmarksetup{startatroot}

\chapter{Xaringan Presentation and Quarto
Book}\label{xaringan-presentation-and-quarto-book}

\bookmarksetup{startatroot}

\chapter{Remote Sensing Image
Corrections}\label{remote-sensing-image-corrections}

\bookmarksetup{startatroot}

\chapter{Remote Sensing Application to Support
Policy}\label{remote-sensing-application-to-support-policy}

\bookmarksetup{startatroot}

\chapter*{References}\label{references}
\addcontentsline{toc}{chapter}{References}

\markboth{References}{References}

\phantomsection\label{refs}
\begin{CSLReferences}{0}{1}
\end{CSLReferences}




\end{document}
